% Options for packages loaded elsewhere
% Options for packages loaded elsewhere
\PassOptionsToPackage{unicode}{hyperref}
\PassOptionsToPackage{hyphens}{url}
\PassOptionsToPackage{dvipsnames,svgnames,x11names}{xcolor}
%
\documentclass[
  letterpaper,
  DIV=11,
  numbers=noendperiod]{scrartcl}
\usepackage{xcolor}
\usepackage{amsmath,amssymb}
\setcounter{secnumdepth}{-\maxdimen} % remove section numbering
\usepackage{iftex}
\ifPDFTeX
  \usepackage[T1]{fontenc}
  \usepackage[utf8]{inputenc}
  \usepackage{textcomp} % provide euro and other symbols
\else % if luatex or xetex
  \usepackage{unicode-math} % this also loads fontspec
  \defaultfontfeatures{Scale=MatchLowercase}
  \defaultfontfeatures[\rmfamily]{Ligatures=TeX,Scale=1}
\fi
\usepackage{lmodern}
\ifPDFTeX\else
  % xetex/luatex font selection
\fi
% Use upquote if available, for straight quotes in verbatim environments
\IfFileExists{upquote.sty}{\usepackage{upquote}}{}
\IfFileExists{microtype.sty}{% use microtype if available
  \usepackage[]{microtype}
  \UseMicrotypeSet[protrusion]{basicmath} % disable protrusion for tt fonts
}{}
\makeatletter
\@ifundefined{KOMAClassName}{% if non-KOMA class
  \IfFileExists{parskip.sty}{%
    \usepackage{parskip}
  }{% else
    \setlength{\parindent}{0pt}
    \setlength{\parskip}{6pt plus 2pt minus 1pt}}
}{% if KOMA class
  \KOMAoptions{parskip=half}}
\makeatother
% Make \paragraph and \subparagraph free-standing
\makeatletter
\ifx\paragraph\undefined\else
  \let\oldparagraph\paragraph
  \renewcommand{\paragraph}{
    \@ifstar
      \xxxParagraphStar
      \xxxParagraphNoStar
  }
  \newcommand{\xxxParagraphStar}[1]{\oldparagraph*{#1}\mbox{}}
  \newcommand{\xxxParagraphNoStar}[1]{\oldparagraph{#1}\mbox{}}
\fi
\ifx\subparagraph\undefined\else
  \let\oldsubparagraph\subparagraph
  \renewcommand{\subparagraph}{
    \@ifstar
      \xxxSubParagraphStar
      \xxxSubParagraphNoStar
  }
  \newcommand{\xxxSubParagraphStar}[1]{\oldsubparagraph*{#1}\mbox{}}
  \newcommand{\xxxSubParagraphNoStar}[1]{\oldsubparagraph{#1}\mbox{}}
\fi
\makeatother

\usepackage{color}
\usepackage{fancyvrb}
\newcommand{\VerbBar}{|}
\newcommand{\VERB}{\Verb[commandchars=\\\{\}]}
\DefineVerbatimEnvironment{Highlighting}{Verbatim}{commandchars=\\\{\}}
% Add ',fontsize=\small' for more characters per line
\usepackage{framed}
\definecolor{shadecolor}{RGB}{241,243,245}
\newenvironment{Shaded}{\begin{snugshade}}{\end{snugshade}}
\newcommand{\AlertTok}[1]{\textcolor[rgb]{0.68,0.00,0.00}{#1}}
\newcommand{\AnnotationTok}[1]{\textcolor[rgb]{0.37,0.37,0.37}{#1}}
\newcommand{\AttributeTok}[1]{\textcolor[rgb]{0.40,0.45,0.13}{#1}}
\newcommand{\BaseNTok}[1]{\textcolor[rgb]{0.68,0.00,0.00}{#1}}
\newcommand{\BuiltInTok}[1]{\textcolor[rgb]{0.00,0.23,0.31}{#1}}
\newcommand{\CharTok}[1]{\textcolor[rgb]{0.13,0.47,0.30}{#1}}
\newcommand{\CommentTok}[1]{\textcolor[rgb]{0.37,0.37,0.37}{#1}}
\newcommand{\CommentVarTok}[1]{\textcolor[rgb]{0.37,0.37,0.37}{\textit{#1}}}
\newcommand{\ConstantTok}[1]{\textcolor[rgb]{0.56,0.35,0.01}{#1}}
\newcommand{\ControlFlowTok}[1]{\textcolor[rgb]{0.00,0.23,0.31}{\textbf{#1}}}
\newcommand{\DataTypeTok}[1]{\textcolor[rgb]{0.68,0.00,0.00}{#1}}
\newcommand{\DecValTok}[1]{\textcolor[rgb]{0.68,0.00,0.00}{#1}}
\newcommand{\DocumentationTok}[1]{\textcolor[rgb]{0.37,0.37,0.37}{\textit{#1}}}
\newcommand{\ErrorTok}[1]{\textcolor[rgb]{0.68,0.00,0.00}{#1}}
\newcommand{\ExtensionTok}[1]{\textcolor[rgb]{0.00,0.23,0.31}{#1}}
\newcommand{\FloatTok}[1]{\textcolor[rgb]{0.68,0.00,0.00}{#1}}
\newcommand{\FunctionTok}[1]{\textcolor[rgb]{0.28,0.35,0.67}{#1}}
\newcommand{\ImportTok}[1]{\textcolor[rgb]{0.00,0.46,0.62}{#1}}
\newcommand{\InformationTok}[1]{\textcolor[rgb]{0.37,0.37,0.37}{#1}}
\newcommand{\KeywordTok}[1]{\textcolor[rgb]{0.00,0.23,0.31}{\textbf{#1}}}
\newcommand{\NormalTok}[1]{\textcolor[rgb]{0.00,0.23,0.31}{#1}}
\newcommand{\OperatorTok}[1]{\textcolor[rgb]{0.37,0.37,0.37}{#1}}
\newcommand{\OtherTok}[1]{\textcolor[rgb]{0.00,0.23,0.31}{#1}}
\newcommand{\PreprocessorTok}[1]{\textcolor[rgb]{0.68,0.00,0.00}{#1}}
\newcommand{\RegionMarkerTok}[1]{\textcolor[rgb]{0.00,0.23,0.31}{#1}}
\newcommand{\SpecialCharTok}[1]{\textcolor[rgb]{0.37,0.37,0.37}{#1}}
\newcommand{\SpecialStringTok}[1]{\textcolor[rgb]{0.13,0.47,0.30}{#1}}
\newcommand{\StringTok}[1]{\textcolor[rgb]{0.13,0.47,0.30}{#1}}
\newcommand{\VariableTok}[1]{\textcolor[rgb]{0.07,0.07,0.07}{#1}}
\newcommand{\VerbatimStringTok}[1]{\textcolor[rgb]{0.13,0.47,0.30}{#1}}
\newcommand{\WarningTok}[1]{\textcolor[rgb]{0.37,0.37,0.37}{\textit{#1}}}

\usepackage{longtable,booktabs,array}
\usepackage{calc} % for calculating minipage widths
% Correct order of tables after \paragraph or \subparagraph
\usepackage{etoolbox}
\makeatletter
\patchcmd\longtable{\par}{\if@noskipsec\mbox{}\fi\par}{}{}
\makeatother
% Allow footnotes in longtable head/foot
\IfFileExists{footnotehyper.sty}{\usepackage{footnotehyper}}{\usepackage{footnote}}
\makesavenoteenv{longtable}
\usepackage{graphicx}
\makeatletter
\newsavebox\pandoc@box
\newcommand*\pandocbounded[1]{% scales image to fit in text height/width
  \sbox\pandoc@box{#1}%
  \Gscale@div\@tempa{\textheight}{\dimexpr\ht\pandoc@box+\dp\pandoc@box\relax}%
  \Gscale@div\@tempb{\linewidth}{\wd\pandoc@box}%
  \ifdim\@tempb\p@<\@tempa\p@\let\@tempa\@tempb\fi% select the smaller of both
  \ifdim\@tempa\p@<\p@\scalebox{\@tempa}{\usebox\pandoc@box}%
  \else\usebox{\pandoc@box}%
  \fi%
}
% Set default figure placement to htbp
\def\fps@figure{htbp}
\makeatother





\setlength{\emergencystretch}{3em} % prevent overfull lines

\providecommand{\tightlist}{%
  \setlength{\itemsep}{0pt}\setlength{\parskip}{0pt}}



 


\KOMAoption{captions}{tableheading}
\makeatletter
\@ifpackageloaded{tcolorbox}{}{\usepackage[skins,breakable]{tcolorbox}}
\@ifpackageloaded{fontawesome5}{}{\usepackage{fontawesome5}}
\definecolor{quarto-callout-color}{HTML}{909090}
\definecolor{quarto-callout-note-color}{HTML}{0758E5}
\definecolor{quarto-callout-important-color}{HTML}{CC1914}
\definecolor{quarto-callout-warning-color}{HTML}{EB9113}
\definecolor{quarto-callout-tip-color}{HTML}{00A047}
\definecolor{quarto-callout-caution-color}{HTML}{FC5300}
\definecolor{quarto-callout-color-frame}{HTML}{acacac}
\definecolor{quarto-callout-note-color-frame}{HTML}{4582ec}
\definecolor{quarto-callout-important-color-frame}{HTML}{d9534f}
\definecolor{quarto-callout-warning-color-frame}{HTML}{f0ad4e}
\definecolor{quarto-callout-tip-color-frame}{HTML}{02b875}
\definecolor{quarto-callout-caution-color-frame}{HTML}{fd7e14}
\makeatother
\makeatletter
\@ifpackageloaded{caption}{}{\usepackage{caption}}
\AtBeginDocument{%
\ifdefined\contentsname
  \renewcommand*\contentsname{Table of contents}
\else
  \newcommand\contentsname{Table of contents}
\fi
\ifdefined\listfigurename
  \renewcommand*\listfigurename{List of Figures}
\else
  \newcommand\listfigurename{List of Figures}
\fi
\ifdefined\listtablename
  \renewcommand*\listtablename{List of Tables}
\else
  \newcommand\listtablename{List of Tables}
\fi
\ifdefined\figurename
  \renewcommand*\figurename{Figure}
\else
  \newcommand\figurename{Figure}
\fi
\ifdefined\tablename
  \renewcommand*\tablename{Table}
\else
  \newcommand\tablename{Table}
\fi
}
\@ifpackageloaded{float}{}{\usepackage{float}}
\floatstyle{ruled}
\@ifundefined{c@chapter}{\newfloat{codelisting}{h}{lop}}{\newfloat{codelisting}{h}{lop}[chapter]}
\floatname{codelisting}{Listing}
\newcommand*\listoflistings{\listof{codelisting}{List of Listings}}
\makeatother
\makeatletter
\makeatother
\makeatletter
\@ifpackageloaded{caption}{}{\usepackage{caption}}
\@ifpackageloaded{subcaption}{}{\usepackage{subcaption}}
\makeatother
\usepackage{bookmark}
\IfFileExists{xurl.sty}{\usepackage{xurl}}{} % add URL line breaks if available
\urlstyle{same}
\hypersetup{
  pdftitle={Complex Disordered Systems},
  colorlinks=true,
  linkcolor={blue},
  filecolor={Maroon},
  citecolor={Blue},
  urlcolor={Blue},
  pdfcreator={LaTeX via pandoc}}


\title{Complex Disordered Systems}
\usepackage{etoolbox}
\makeatletter
\providecommand{\subtitle}[1]{% add subtitle to \maketitle
  \apptocmd{\@title}{\par {\large #1 \par}}{}{}
}
\makeatother
\subtitle{Static and dynamic correlations}
\author{}
\date{}
\begin{document}
\maketitle


\subsection{Today}\label{today}

\begin{itemize}
\tightlist
\item
  \textbf{Characterisation of colloids}

  \begin{itemize}
  \tightlist
  \item
    Static (pair) corrleations
  \item
    Diffusive dynamics
  \end{itemize}
\end{itemize}

\subsection{Pairwise interactions}\label{pairwise-interactions}

True colloidal systems can be complex:

\begin{itemize}
\tightlist
\item
  charges
\item
  shape deformability/swelling
\item
  instabilities of the components
\end{itemize}

Their theoretical description focuses on \textbf{essential ingredients}.

A simplifying assumption is that the potential energy of a collection of
colloids is purely \textbf{pairwise}

\[
U_N\left(\mathbf{r}^N\right)=\sum_{i=1}^{N-1} \sum_{j=i+1}^N V\left(r_i j\right)=\frac{1}{2} \sum_{i \neq j} V\left(r_{i j}\right)
\]

where \(V(r_{ij})\) is the pairwise, radial interaction potential that
only depends on the distance between particle centres
\(r_{ij} = \mathbf{r}_i-\mathbf{r}_j\).

\subsection{Aside: Not pairwise?}\label{aside-not-pairwise}

Some examples of non-pariwise situations:

\begin{itemize}
\tightlist
\item
  \textbf{Atomistic models}: Electronic densities that fluctuate due to
  quantum effects are intrinsically many-body
\item
  \textbf{True colloid-cpolymer mixtures}: if the polymers are not
  ideal, their spatial distribution and their forces depend on the
  positions of more than two colloids
\end{itemize}

\begin{figure}[H]

{\centering \includegraphics[width=\linewidth,height=4.6875in,keepaspectratio]{./figs/dft-nonideal-colloid-polymer.png}

}

\caption{\(C_{60}\) isosurface via density functional theory (Wikipedia)
and configuration of non-deal colloid-polymer mixture, Rohrbach et al
JCP (2022)}

\end{figure}%

\subsection{Pairwise interactions and pair
correlations}\label{pairwise-interactions-and-pair-correlations}

\begin{itemize}
\tightlist
\item
  For \textbf{pairwise} systems, all the thermodynamic properties are
  solely encoded in the \textbf{pairwise correlations}.
\end{itemize}

You have already seen such correlations in other contexts (Ising model,
lattice gas).

The key pair correlation is the real space \textbf{radial distribution
function}:

\[
g(r)=\frac{1}{\rho N}\left\langle\sum_{i=1}^N \sum_{j \neq i} \delta\left(\left|\mathbf{r}_i-\mathbf{r}_j\right|-r\right)\right\rangle
\]

The free energy can then be written as \[
\frac{F_{\mathrm{ex}}}{k_B T}=2 \pi \rho N \int_0^{\infty}[g(r) \ln g(r)-g(r)+1] r^2 d r+\frac{\rho N}{2 k_B T} \int V(r) g(r) d^3 r
\]

and you should be able to read out a part very reminiscent of an
(information-theoretical) entropy and one reminiscent of an internal
energy.

\subsection{Radial distribution
function}\label{radial-distribution-function}

\begin{figure}[H]

{\centering \pandocbounded{\includegraphics[keepaspectratio]{./figs/gr.png}}

}

\caption{The radial distribution function, algorithmic interpretation:
at a distance \(r\), we count the number of particle centres within a
slice of width dr and then normalise.}

\end{figure}%

\subsection{Radial distribution
function}\label{radial-distribution-function-1}

\begin{itemize}
\item
  The radial distribution function \(g(r)\) is related to the average
  local density at a distance \(r\) from a reference particle:
\item
  To formally derive this, consider a system of \(N\) particles in
  volume \(V\) with average density \(\rho = N/V\). The local density at
  a distance \(r\) from a reference particle is defined as \[
  \langle \rho(r) \rangle = \left\langle \sum_{j \neq i} \delta\left(|\mathbf{r}_i - \mathbf{r}_j| - r\right) \right\rangle
  \] Averaging over all particles and normalizing by the bulk density
  \(\rho\) gives the radial distribution function: \[
  g(r) = \frac{1}{\rho N} \left\langle \sum_{i=1}^N \sum_{j \neq i} \delta\left(|\mathbf{r}_i - \mathbf{r}_j| - r\right) \right\rangle = \frac{\langle \rho(r) \rangle}{\rho}
  \]

  where \(\langle \rho(r) \rangle\) is the average density at distance
  \(r\) from a particle, and \(\rho\) is the bulk density.
\end{itemize}

\subsection{Radial distribution
function}\label{radial-distribution-function-2}

\begin{itemize}
\tightlist
\item
  For an \textbf{ideal gas}, \(g(r) = 1\) everywhere (no correlations).
\item
  For \textbf{hard spheres}, \(g(r) = 0\) for \(r < \sigma\) (no
  overlap), and \(g(r)\) shows oscillations at higher \(r\) due to
  packing effects.
\end{itemize}

\begin{tcolorbox}[enhanced jigsaw, opacityback=0, coltitle=black, titlerule=0mm, colback=white, left=2mm, opacitybacktitle=0.6, colframe=quarto-callout-note-color-frame, title=\textcolor{quarto-callout-note-color}{\faInfo}\hspace{0.5em}{Note}, colbacktitle=quarto-callout-note-color!10!white, leftrule=.75mm, toptitle=1mm, rightrule=.15mm, toprule=.15mm, arc=.35mm, bottomrule=.15mm, bottomtitle=1mm, breakable]

Hard spheres in 3D are not exactly solvable! We only have perturbative
solutons to hard spheres

\end{tcolorbox}

\begin{itemize}
\tightlist
\item
  The \(g(r)\) is the central object of (equilibrium) liquid state
  theory
\item
  Correlations between two particles \(1\) and \(2\) in a fluid to have
  two possible origins

  \begin{itemize}
  \tightlist
  \item
    a \textbf{direct} correlation between the two particles, mediated by
    direct interactions (e.g.~collisions) between 1 and 2.
  \item
    an \textbf{indirect correlation}, mediated by other particles in the
    fluid
  \end{itemize}
\end{itemize}

\subsection{Direct and indirect
correlations}\label{direct-and-indirect-correlations}

\begin{itemize}
\tightlist
\item
  The \(g(r)\) contains \textbf{both}.
\item
  We can assume the existence of a correlation function that instead
  only incl;udes the \textbf{direct} part and call it \(c(r)\). Then a
  hierarchy exists
\end{itemize}

\[
h\left(r_{12}\right)=c\left(r_{12}\right)+\rho \int c\left(r_{13}\right) h\left(r_{32}\right) d \mathbf{r}_3
\]

where \(h(r) = g(r)-1\). This is the \textbf{Ornstein-Zernicke integral
equation} between particles 1,2 and 3.

Both \(h(r)\) and \(c(r)\) are unknwon in principle. One needs an
additional condition to fix box: these are \textbf{physical closures}
leading to approximations.

\begin{figure}[H]

{\centering \pandocbounded{\includegraphics[keepaspectratio]{./figs/ornstein-zernicke.png}}

}

\caption{Meaning of teh Ornstein-Zernicke relation, adapted from Santos,
Springer(2016)}

\end{figure}%

\subsection{Percus-Yevick
approximation}\label{percus-yevick-approximation}

A common closure is the \textbf{Percus-Yevick approximation}.

\[
c(r)=[1+h(r)]\left[1-e^{\beta U(r)}\right]
\]

This closure interpolates between the hard-core exclusion (for
\(U(r) \to \infty\), \(c(r) \to 0\) inside the core) and the ideal gas
limit (for \(U(r) = 0\), \(c(r) = 0\)).

The result (semi) analytical. For hard spheres, it allows to calculate
the g(r) accurately in a wide range of packing fractions

\begin{figure}[H]

{\centering \includegraphics[width=\linewidth,height=3.64583in,keepaspectratio]{./figs/hs47.png}

}

\caption{PY approximation vs simulated data for hard spheres.}

\end{figure}%

\subsection{\texorpdfstring{Equation of states from
\(g(r)\)}{Equation of states from g(r)}}\label{equation-of-states-from-gr}

The knowledge of the pair correlations also allows one to extract
equation of states and significantly improve on the virial expansion, as
the \(g(r)\) contains packing effects!

\[
P=\rho k_B T-\frac{2 \pi \rho^2}{3} \int_0^{\infty} r^3 \frac{d u(r)}{d r} g(r) d r
\]

\begin{tcolorbox}[enhanced jigsaw, opacityback=0, coltitle=black, titlerule=0mm, colback=white, left=2mm, opacitybacktitle=0.6, colframe=quarto-callout-note-color-frame, title=\textcolor{quarto-callout-note-color}{\faInfo}\hspace{0.5em}{Note}, colbacktitle=quarto-callout-note-color!10!white, leftrule=.75mm, toptitle=1mm, rightrule=.15mm, toprule=.15mm, arc=.35mm, bottomrule=.15mm, bottomtitle=1mm, breakable]

For hard spheres, The derivative \(\frac{du(r)}{dr}\) is zero everywhere
except at the contact point \(r = \sigma\), where it becomes a delta
function: \[
\frac{dV(r)}{dr} = -\infty \delta(r - \sigma)
\]

Substituting into the pressure equation: \[
P = \rho k_B T + \frac{2\pi\rho^2}{3} \sigma^3 g(\sigma)
\]

Therefore, for hard spheres, the equation of state only requires
knowledge of the contact value \(g(\sigma)\).

We will see this in a problem class.

\end{tcolorbox}

\subsection{Structure factor}\label{structure-factor}

The real-space correlations and relations have their reciprocal
(\emph{Fourier} space) counterparts.

The structure factor is defined as

\[
S(k)=1+\rho \int[g(r)-1] e^{-i \mathbf{k} \cdot \mathbf{r}} d \mathbf{r}
\]

Isotropicity in 3D simplifies the expression to

\[
S(k)=1+4 \pi \rho \int_0^{\infty} r^2[g(r)-1] \frac{\sin (k r)}{k r} d r
\]

\begin{itemize}
\tightlist
\item
  Radial distribution functions \(\to\) easily accessed in
  \textbf{confocal experiments} (large colloids)
\item
  Structure factors \(\to\) immediately accessed via \textbf{scattering
  experiments}
\end{itemize}

\subsection{Structure factor}\label{structure-factor-1}

Interestingly, reciprocal space simplifies integral equations such as
Ornstein-Zernicke

\[
\tilde{h}(k)=\frac{\tilde{c}(k)}{1-\rho \tilde{c}(k)}
\]

allowing one to read the direct correlation function directly from
scattering data, as \(\tilde{h}(k)=\) can be identified with the
structure factor.

\begin{figure}[H]

{\centering \includegraphics[width=\linewidth,height=4.6875in,keepaspectratio]{./figs/gallium.png}

}

\caption{Structure factors (left) and radial distribution functions
(right) for Gallium at extreme conditions, Drewitt et al PRL (2020)}

\end{figure}%

\subsection{Aside: higher order
correlations}\label{aside-higher-order-correlations}

Nothing prevents us to quantify higher order correlations. In general
these are non-trivial are controlled by the indirect part of the
correlations.

Standard ways to identify \textbf{n-particle}* correlations is to
measure \textbf{structural motifs}.

\begin{figure}[H]

{\centering \pandocbounded{\includegraphics[keepaspectratio]{./figs/josh.png}}

}

\caption{Many body correlations in hard spheres, seminal work from Josh
Robinson, PhD student in Bristol a few years ago, Robinson et al PRL
(2019)}

\end{figure}%

This is particularly important for dense systems (e.g.~glasses, see
future lecture).

\subsection{Dynamics: single vs collective
displacements}\label{dynamics-single-vs-collective-displacements}

The thermal motion of individual colloids is often well modeled via the
\textbf{Langevin equation}

\[
m \frac{d \vec{v}}{d t}=-\gamma \vec{v}+\vec{\eta}(t)
\]

Coarse-graining over small volumes (and short times) gives us access to
the local density \(\rho(\mathbf{r},t)\) which obeys a \textbf{diffusion
equation}

\[
\frac{\partial \rho(\mathbf{r}, t)}{\partial t}=D \nabla^2 \rho(\mathbf{r}, t)
\]

The diffusivity is linked to the microscopic Langevin parameters via
\textbf{fluctuation-dissipation} (Einstein's relation)

\[
D=\frac{k_B T}{\gamma}
\]

(where the friction is the dissipation and the random force is the
fluctuations).

\subsection{Diffusion equation: macroscopic
picture}\label{diffusion-equation-macroscopic-picture}

It is possible to (re)-derive the diffusion equation solely from
macroscopic requirements. We have a macroscopic density \(\rho(r,t)\)
and a macroscpic flux \(\vec{J}(r,t)\)

\begin{itemize}
\tightlist
\item
  \textbf{continuity}: there is no mass loss in the dynamics
\end{itemize}

\[
\frac{\partial \rho(\mathbf{r}, t)}{\partial t}+\nabla \cdot \mathbf{J}(\mathbf{r}, t)=0
\] where the divergence \(\nabla \cdot \mathbf{J}(\mathbf{r}, t)\)
represents the net outflow of particles from a given region due to the
flux \(\mathbf{J}\).

\begin{itemize}
\tightlist
\item
  \textbf{Fick's law} : we assume the flow is a gradient field of the
  density, flowing from high to low density
\end{itemize}

\[
\mathbf{J}(\mathbf{r}, t)=-D \nabla \rho(\mathbf{r}, t)
\]

It is immediate to combine the two to obtain the diffusion equation

\[
\frac{\partial \rho(\mathbf{r}, t)}{\partial t}=D \nabla^2 \rho(\mathbf{r}, t)
\]

\subsection{Measuring diffusion
coefficients}\label{measuring-diffusion-coefficients}

The distribution \(\rho(\mathbf{r},t)\) encodes all the information
about the motion of an ensemble.

Often, less is sufficient to characterise the diffusion. The second
moment of the displacements (the mean squared displacement) provides
this information. In \(d\) dimensions \[
\left\langle \left| \mathbf{r}(t) - \mathbf{r}(t_0) \right|^2 \right\rangle = 2 d D t
\]

Hence \(D\) can be extracted from linear fits to the MSD.

\subsection{Stokes-Einstein relation: fluctuation dissipation for
colloids}\label{stokes-einstein-relation-fluctuation-dissipation-for-colloids}

In dilute conditions, let's think of a particle slowly sedimenting in a
viscous fluid.

The density distribution is Boltzmann

\[
n(x)=n_0 \exp \left(-U(x) / k_B T\right)=n_0 \exp \left(-F x / k_B T\right)
\]

where \(F= m_B g\) is the force from the buoyant mass.

A flux is produced \(J_F = n(x) v=\frac{n(x) F}{\xi}\)

where we used the fact that the velocity is set by the viscosity
\(\eta\) as \(v=F/\eta\).

Diffusion occurs at the same time, driven by the gradient (Fick's law)
\[
J_D(x)=-D \frac{\partial n(x)}{\partial x}
\]

\subsection{Stokes-Einstein relation: fluctuation dissipation for
colloids}\label{stokes-einstein-relation-fluctuation-dissipation-for-colloids-1}

Equating the two we have

\[
\frac{n(x) F}{\xi}=-D \frac{\partial n(x)}{d x}=D \frac{F}{k_B T} n(x)
\]

and differentiating the Boltzmann distribution gives \[
\frac{n(x) F}{\xi}= D \frac{F}{k_B T} n(x)
\]

Hence a balance exists between thermal fluctuations and viscosities. For
a sphere, \textbf{Stoke's law} \(\xi=6 \pi \eta R\) yielding simply

\[
D=\frac{k_B T}{\xi}=\frac{k_B T}{6 \pi \eta R}
\]

This is a re-statement of the Einstein relation, called the
\textbf{Stokes-Einstein} relation.

\[
\text { Fluctuations: } k_B T \quad \longrightarrow \quad \text { Observed motion: } D \quad \longrightarrow \quad \text { Dissipation: } 6 \pi \eta R .
\]

Which is generic under (moderate) driving.

\subsection{Random walkers in 2d,
example}\label{random-walkers-in-2d-example}

\begin{Shaded}
\begin{Highlighting}[]
\NormalTok{\#| echo: false}
\NormalTok{import matplotlib.pyplot as plt}

\NormalTok{import numpy as np}


\NormalTok{def plot\_tj(positions, msd, walker=0, loglog=False):}
\NormalTok{    \# Plot}
\NormalTok{    fig, ax = plt.subplots(1, 3, figsize=(9.5,2.8))}
\NormalTok{    ax[0].plot(positions[0, :, 0], positions[0, :, 1], alpha=0.7, linewidth=0.8)  \# example trajectory}
\NormalTok{    time = np.arange(len(msd))}
\NormalTok{    ax[1].plot(time, msd, \textquotesingle{}o{-}\textquotesingle{}, markersize=3)}
\NormalTok{    \# Fit linear line to MSD data}
\NormalTok{    slope, intercept = np.polyfit(time, msd, 1)}
\NormalTok{    fit\_line = slope * time + intercept}
\NormalTok{    D = slope/4}
\NormalTok{    ax[1].plot(time, fit\_line, \textquotesingle{}r{-}{-}\textquotesingle{}, alpha=0.8, label=f\textquotesingle{}All data: D=\{D:.2f\}\textquotesingle{})}

\NormalTok{    \# partial data}
\NormalTok{    slope, intercept = np.polyfit(time[:len(time)//2], msd[:len(time)//2], 1)}
\NormalTok{    fit\_line = slope * time + intercept}
\NormalTok{    D = slope/4}
\NormalTok{    ax[1].plot(time, fit\_line, \textquotesingle{}g{-}{-}\textquotesingle{}, alpha=0.8, label=f\textquotesingle{}Half data: D=\{slope:.2f\}\textquotesingle{})}

\NormalTok{    ax[1].legend(fontsize=8, frameon=False)}
\NormalTok{    ax[1].set(xlabel="time",ylabel="MSD")}
\NormalTok{    if loglog:}
\NormalTok{        ax[1].set\_xscale(\textquotesingle{}log\textquotesingle{})}
\NormalTok{        ax[1].set\_yscale(\textquotesingle{}log\textquotesingle{})}

\NormalTok{    bins = np.arange({-}200,200,20.0)}
\NormalTok{    H, edgesx, edgesy = np.histogram2d(positions[:,{-}1,0],positions[:,{-}1,1],bins=(bins,bins))}


\NormalTok{    ax[2].imshow(H.T)}
\NormalTok{    ax[0].axis("equal")}
\NormalTok{    ax[2].axis("equal")}
\NormalTok{    ax[2].axis("off")}
\NormalTok{    plt.tight\_layout()}
\NormalTok{    plt.show()}
\end{Highlighting}
\end{Shaded}

\begin{Shaded}
\begin{Highlighting}[]
\NormalTok{n\_steps, n\_walkers, dim = 10000, 500, 2}
\NormalTok{steps = np.random.randint({-}1, 2, size=(n\_walkers, n\_steps, dim))}
\NormalTok{positions = np.cumsum(steps, axis=1)  \# cumulative sum along steps}
\NormalTok{msd = positions.var(axis=0).sum(axis=1)  \# MSD at each timestep}
\NormalTok{plot\_tj(positions,msd,walker=0,loglog=False)}
\end{Highlighting}
\end{Shaded}

\subsection{Hidden assumptions}\label{hidden-assumptions}

\begin{tcolorbox}[enhanced jigsaw, opacityback=0, coltitle=black, titlerule=0mm, colback=white, left=2mm, opacitybacktitle=0.6, colframe=quarto-callout-note-color-frame, title=\textcolor{quarto-callout-note-color}{\faInfo}\hspace{0.5em}{Note}, colbacktitle=quarto-callout-note-color!10!white, leftrule=.75mm, toptitle=1mm, rightrule=.15mm, toprule=.15mm, arc=.35mm, bottomrule=.15mm, bottomtitle=1mm, breakable]

\begin{itemize}
\tightlist
\item
  We have been considering ensembles of particles microscopically
  governed by the Langevin equation
\item
  We did not comment on the inter-particle interactions and assumed that
  the properties of the thermal bath determine the diffusivity \(D\). In
  reality, interactions between particles can lead to deviations from
  simple diffusion, especially at higher concentrations, where
  collective effects and hydrodynamic interactions become important.
\end{itemize}

\end{tcolorbox}

\subsection{Dilute vs dense diffusion}\label{dilute-vs-dense-diffusion}

\begin{itemize}
\item
  Consider hard spheres in a fluid.
\item
  At low volume fractions \(\phi\) they rarely collide and they perofrm
  under/over-damped (Langevin) dynamics. They will therefore diffuse as
  \(\operatorname{MSD} (t)= 6D_0 t\)
\item
  Things change when we increase the packing.

  \begin{itemize}
  \tightlist
  \item
    Short times: \(D_s <D_0\), diffusion within \textbf{cages} formed by
    the neighbours. Mostly free, but with some collisions.
  \item
    Long times: \(D_L\ll D_0\) cage breaking. Interactions are both hard
    core and hydrodynamic
  \end{itemize}
\end{itemize}

\begin{figure}[H]

{\centering \includegraphics[width=\linewidth,height=4.47917in,keepaspectratio]{./figs/colloids-msd.png}

}

\caption{MSD of colloids}

\end{figure}%

\subsection{Dilute vs dense
diffusion}\label{dilute-vs-dense-diffusion-1}

\begin{itemize}
\tightlist
\item
  At \textbf{very high} volume fractions the cage mechanism becomes more
  complex: paricles are trapped for long times and only occasionaly
  break cages:

  \begin{itemize}
  \tightlist
  \item
    cage-jump dynamics \(\to\) glassy dynamics (see future lectures)
  \item
    rare events \(\to\) Levy-flight statistics \(\neq\)
    Gaussian/diffusive statistics
  \item
    Breaking of fluctuation-dissipation condition encoded in the
    \textbf{Stokes-Einstein} relation

    \begin{itemize}
    \tightlist
    \item
      the viscosity increases much faster than the decrease in diffusion
      \(D\xi/T\neq \mathrm{constant}\)
    \end{itemize}
  \end{itemize}
\end{itemize}




\end{document}
